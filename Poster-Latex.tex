\documentclass{article}
\usepackage{lmodern}
\usepackage[T1]{fontenc}
\usepackage[english,activeacute]{babel}
\usepackage{mathtools}
\usepackage{biblatex}
\usepackage{csquotes}

\title{User Behavior Analysis in Campus Area Networks through Kohonen Self Organizing Feature Maps}

\author{Nelson Victor Cruz Hern\'andez }

\date{December 2017}

\begin{document}

\maketitle

\section{Abstract}
This work presents a novel implmentation for Kohonen Self Organizing Feature Maps (SOM) suitable for user behavior analysis in Campus Area Networks (CAN). SOM algorithm works under competitive and unsupervised learning approach, making it perfect for clustering and feature selection tasks. SOM is used to create behavior clusters through the analysis of user network traffic, in fixed time intervals. To demonstrate the feasibility of SOM, resulting users clusters are merged and evaluated against new matching vectors. Experimental results, show that an user can be identified based on it's network traffic since it generates a match in a previous created cluster.

\section{Introduction}
A network intrusion attack can be any use of the network that compromises its stability or the security of the information that is stored on devices connected to it [1]. The weakest element in network security is an user, that may act as an active or passive attacker [2]. User behavior is a wide open studied topic using Machine Learning Classification and Clustering algorithms [3 - 6], focused on detecting popularity of applications, user session duration, and users distribution across access points [7]. This work makes two assumptions: It is possible to identify an user through it's network traffic data and user normal network traffic differs from an attacker network traffic. Self Organizing Maps cluster algorithm is used to obtain a pattern of user behavior. Obtained patterns define specific user behavior and mark differences between normal and attacker users.
\section{Kohonen Self Organizing Maps}
\section{User behavior with Kohonen Self Organizing Maps}
\section{User recognition results}
\section{Conclusions}
\section{References}

\end{document}