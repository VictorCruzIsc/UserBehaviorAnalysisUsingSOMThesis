\documentclass{article}
\usepackage{lmodern}
\usepackage[T1]{fontenc}
\usepackage[english,activeacute]{babel}
\usepackage{mathtools}
\usepackage{biblatex}
\usepackage{csquotes}

\title{User Behavior Analysis in Campus Area Networks through Kohonen Self Organizing Feature Maps}

\author{Nelson Victor Cruz Hern\'andez }

\date{May 2017}

\begin{document}

\maketitle

\section{Introduction}

\subsection{Antecedentes}

\subsection{Justification}

\subsection{Problem}

\subsection{Hypothesis}

\subsection{Objectives}

\subsubsection{General}

\subsection{Particular}

\section{State of the Art}

\subsection{Machine Learning Algorithms y Seguridad Inform�tica}

\subsection{Profiling / User clasification}

\section{Marco Te�rico}

\subsection{Redes de Computadoras}

\subsubsection{Network topology}

\subsubsection{OSI Model / TCPIP}

\subsubsection{Network Security}

\subsubsection{Intrussion Detection Systems}

\subsection{Machine Learning Algorithms}

\section{Desarrollo Metodol�gico}

\subsection{Experiment context}
Experiment was carried out on a Campus Area Network (CAN) that has a 16-bit network and a Windows domain controller, using a HTTP proxy. Among campus applications web and remote apps are included. Email service is provided by Microsoft Exchange Server which is hosted outside the campus network.
The target users were full-time professors who had a computer with a static IP address and a wireless access with a dynamic IP address.
Five full-time professors (hereafter denoted as users). were selected for the experiment. For each one, real traffic was captured (K1 to K5) during a week, and then processed in order to get its individual behavior matrix. Once each user has an individual matrix, this was compared against himself and other user matrix, in order to know if the user has a similar behavior, or recognizes itself accurately.

\subsection{Explanation}
Self Organized Maps (SOM) algorithm works as an unsupervised learning clustering approach, where training is entirely data-driven and no target results for the input data vectors are provided. It also provides a topology to preserve mapping from high dimensional space to nodes that form a two-dimensional lattice that represents high dimensional space onto a plane in which map units are grouped by it's features values similarity. Each node has a specific topological position and contains a vector weights (features) of the same dimension as the input vector [8].

Learning algorithm of Conventional SOM
1) Initialize the map using random vectors.

2) Searching for the winner unit
Select an input vector x randomly from learning set.
Search for the unit ???? which is associated to the closest vector ???? to x which minimize the quantization error |x ? m|.

3) Updating the winner unit and its neighboring units.
For the winner units ???? and its neighbor U ? w, update the vectors associated to the units using the following equation.
????=????+?????? �?� ???????
where ???? ?? is neighborhood function which is the decreasing function of distance d between ???? and ?????
and ? is the learning rate.

4) Repeat Step 2, Step 3 with decreasing neighborhood function ???? ?? and learning rate ? until the quantization error converges enough or during the pre-defined iterations

\subsection{Data collecting}

\subsection{Experiment execution}

\subsubsection{Capturing regular traffic}
How data was captured

\subsubsection{Processing raw data}
Data Chunck creation
Data Set Creation (Lattice creation, Train, Evaluation)

\subsubsection{Creating user behavior lattice}
Feature selection
SOM Training

\subsubsection{Comparing user behavior lattices}
Comparison between two different lattices of the same user
Comparison between different lattices of multiple users

\section{Results and Discussion}
Results presentation, how the results are interpreted, and what we can do with data

\section{Conclusions}

\subsection{Conclusions}

\subsection{Future work}

\section{Bibliography}
[8] Dozono, H., Itou, S., and Nakakuni, M. (2007). Comparison of the adaptive authentication systems for behavior biometrics using the variations of self organizing maps. International Journal of Computers and Communications, 1(4), 108-116.

\end{document}